\documentclass{article}
\usepackage{tocloft}
\include{common_symbols_and_format}
\renewcommand{\cfttoctitlefont}{\Large\bfseries}

\begin{document}
\logo
\rulename{Percentage Price Oscillator} %Argument is name of rule
\tblofcontents

\ruledescription{The percentage price oscillator (PPO) is a momentum indicator that measures the difference between two moving averages as a percentage of the larger moving average. The moving averages are a 26-period and 12-period exponential moving average (EMA). The Percentage Price Oscillator is shown with a signal line, a histogram and a centerline. Signals are generated with signal line crossovers, centerline crossovers, and divergences. A bullish reversal of an asset is identify when the PPO cross above zero line.
                 And a bearish reversal when the PPO cross below the zero line.
}

\howtotrade
{The strategy is to identify asset's price cycles.
Bullish Reversal - when PPO is above zero \&
Bearish Reversal - when PPO is below zero.
}

\ruleparameters %You can include however many arguments (in groups of 4) as you want!
{Short term look back Length}{12}{Short term look back length used to compute EMA.}{$\lookbacklength_{s}$}
{Long term look back Length}{26}{Long term look back length used to compute EMA.}{$\lookbacklength_{l}$}
{Signal look back Length}{9}{Look back length used to generate Signal line.}{$S_{l}$}
\stoptable %must be included or Tex engine runs infinitely

\newpage
\section{Equation}
Below are the equations which govern how this specific trading rule calculates a trading position.

\begin{equation}
    PPO = \frac{EMA(\lookbacklength_{s}) -         EMA(\lookbacklength_{l})}{EMA(\lookbacklength_{l})} \times 100
\end{equation}
\begin{equation}
    Signal = EMA(S_{l})
\end{equation}
\\ % creates some space after equation
where:

$EMA(\lookbacklength_{s})$: is the short term exponentially weighted average.

$EMA(\lookbacklength_{l})$: is the long term exponentially weighted average.

$EMA(S_{l})$: is the exponentially weighted average computed to generate signal line.

\keyterms
\furtherlinks %The footer
\end{document}