\documentclass{article}
\usepackage{tocloft}
\include{common_symbols_and_format}
\renewcommand{\cfttoctitlefont}{\Large\bfseries}

\begin{document}
\logo
\rulename{Aroon} %Argument is name of rule
\tblofcontents

\ruledescription{The aroon indicator developed by Tushar Chande in 1995, is a technical indicator used to identify trend
                changes. It can also reveal the start of a new trend, its strength and can help identify sideway trends.
                It is consist of two indicators, the Aroon up and the Aroon Down.
                The Aroon up reflects the number of days since the most recent 25-day high.
                The Aroon down indicator shows the number of days since the most recent 25-day low.
                This two are also called bullish Aroon and bearish Aroon.}

\howtotrade
{The strategy is to identify trend shifts or reversals.
Bullish Reversal - when Aroon up crossover above the Aroon down.
Bearish Reversal - when Aroon up crossover below the Aroon down.
}

\ruleparameters %You can include however many arguments (in groups of 4) as you want!
{High Periods}{25}{Periods Since \textit{\textbf{n}} period High}{$P_h$}
{Low Periods}{25}{Periods Since \textit{\textbf{n}} period Low}{$P_l$}
\stoptable %must be included or Tex engine runs infinitely


\section{Equation}
Below are the equations which govern how this specific trading rule calculates a trading position.

\begin{equation}
A_u = \frac{25 - P_h}{25} \times 100
\end{equation}

\begin{equation}
A_d = \frac{25 - P_l}{25} \times 100
\end{equation}


with:

$A_u$: is the AROON Up.

$A_d$: is the AROON Down.

$P_h$: Periods Since 25 period High.

$P_u$: Periods Since 25 period Low.


\keyterms
\furtherlinks %The footer
\end{document}