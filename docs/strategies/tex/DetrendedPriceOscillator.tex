\documentclass{article}
\usepackage{tocloft}
\include{common_symbols_and_format}
\renewcommand{\cfttoctitlefont}{\Large\bfseries}

\begin{document}
\logo
\rulename{Detrended Price Oscillator} %Argument is name of rule
\tblofcontents

\ruledescription{The Detrended Price Oscillator (DPO) is a lagging indicator that attempts to filter out short-term price trends to highlight the broader cycle of price movements. . To accomplish this, the moving average (generally a 14-period) becomes a straight line and price variation above and below the moving average becomes the Price Oscillator.
                A bullish trend shift is indicated when the DPO crosses above zero and a bearish trend shift if the DPO crosses below zero.}

\howtotrade
{The strategy is to identify asset's price cycles.
Bullish Reversal - when DPO is above zero \&
Bearish Reversal - when DPO is below zero.
}

\ruleparameters %You can include however many arguments (in groups of 4) as you want!
{Look Back}{20}{Number of look back periods.}{\lookbacklength}
{Simple Moving Average}{20}{Simple Moving Average of \textit{\textbf{n}} periods.}{$SMA(\lookbacklength)$}
\stoptable %must be included or Tex engine runs infinitely


\section{Equation}
Below are the equations which govern how this specific trading rule calculates a trading position.

\begin{equation}
DPO = P_{\frac{\lookbacklength}{2} + 1} - SMA(\lookbacklength)
\end{equation}
\\
where: \\
$P_{\frac{\lookbacklength}{2} + 1}$ : Closing Price of asset where look back periods are determined using $\frac{\lookbacklength}{2} + 1$.
\\ \\
e.g. if $\lookbacklength = 20$ then closing price of asset would be from $\frac{20}{2} +1 = 11$ periods ago.
\\ \\
$SMA(\lookbacklength)$ : is the Simple Moving Average with number of periods determined using parameter \lookbacklength.


\keyterms
\furtherlinks %The footer
\end{document}