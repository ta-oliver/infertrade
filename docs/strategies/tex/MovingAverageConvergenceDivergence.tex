\documentclass{article}
\usepackage{tocloft}
\include{common_symbols_and_format}
\renewcommand{\cfttoctitlefont}{\Large\bfseries}

\begin{document}
\logo
\rulename{Moving Average Convergence Divergence} %Argument is name of rule
\tblofcontents

\ruledescription{Moving Average Convergence Divergence (MACD), developed by Gerald Appel towards the end of 1970s is a momentum indicator that is mostly used by traders to identify trend shifts. This indicator is used to understand the momentum and its directional strength by calculating the difference between two time period intervals, which are a collection of historical time series. A Bullish momentum is identified when the MACD line cross above the signal line.
                A bearish momentum is identified when the MACD line cross below the signal line.
}

\howtotrade
{The strategy is to identify bullish and bearish crossovers.
Bullish crossover - when MACD line cross above the signal line
Bearish Crossover - when signal line cross above the MACD line.
}

\ruleparameters
{Short term look back Length}{12}{Short term look back length used to compute EMA.}{$\lookbacklength_{s}$}
{Long term look back Length}{26}{Long term look back length used to compute EMA.}{$\lookbacklength_{l}$}
{Signal look back Length}{9}{Look back length used to generate Signal line.}{$S_{l}$}
\stoptable

\newpage
\section{Equation}
Below are the equations which govern how this specific trading rule calculates a trading position.

\begin{equation}
MACD = EMA(\lookbacklength_{s}) - EMA(\lookbacklength_{l})
\end{equation}
\\
\begin{equation}
Signal = EMA(S_{l})
\end{equation}
\\ % creates some space after equation
with:

$EMA(\lookbacklength_{s})$: is the short term exponentially weighted average.

$EMA(\lookbacklength_{l})$: is the long term exponentially weighted average.

$EMA(S_{l})$: is the exponentially weighted average computed to generate signal line.


\keyterms
\furtherlinks %The footer
\end{document}
