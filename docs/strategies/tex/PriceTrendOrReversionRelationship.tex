\documentclass{article}%
\usepackage[T1]{fontenc}%
\usepackage[utf8]{inputenc}%
\usepackage{lmodern}%
\usepackage{textcomp}%
\usepackage{lastpage}%
\usepackage{relsize}%
\usepackage{ragged2e}%
\usepackage{tocloft}
%

\usepackage[utf8]{inputenc}
\usepackage{amsmath}
\usepackage{hyperref}
\usepackage{array}
\usepackage{graphicx}
\usepackage{longtable}
\usepackage{tikz}
\usepackage{titlesec}
\usepackage{subcaption}
\usepackage[margin=1in]{geometry}
\usepackage{float}
\usepackage{multicol}
\usepackage{hyperref}
\usepackage{url}

\newcommand{\position}{z}
\newcommand{\research}{r}
\newcommand{\price}{p}
\newcommand{\observed}{o}
\newcommand{\regressionprice}{y}
\newcommand{\currenttime}{t}
\newcommand{\constantc}{c}
\newcommand{\minimumratio}{K}
\newcommand{\lookbacklength}{L}
\newcommand{\dummytime}{\tau}
\newcommand{\dummyiterator}{n}
\newcommand{\dummyiteratortwo}{m}
\newcommand{\dummyiteratorthree}{l}
\newcommand{\decaycoefficient}{\phi}
\newcommand{\decaycoefficientone}{\decaycoefficient_{1}}
\newcommand{\decaycoefficienttwo}{\decaycoefficient_{2}}
\newcommand{\decaycoefficientthree}{\decaycoefficient_{3}}
\newcommand{\numberofcoefficients}{\Phi}
\newcommand{\fixedconstant}{\Omega}
\newcommand{\amplitudecoefficient}{\kappa}
\newcommand{\amplitudecoefficientone}{\amplitudecoefficient_{1}}
\newcommand{\amplitudecoefficienttwo}{\amplitudecoefficient_{2}}
\newcommand{\amplitudecoefficientthree}{\amplitudecoefficient_{3}}
\newcommand{\horizon}{h}
\newcommand{\contribution}{\lambda}
\newcommand{\contributionone}{\contribution_{1}}
\newcommand{\contributiontwo}{\contribution_{2}}
\newcommand{\bigcontribution}{\Lambda}
\newcommand{\bigcontributionone}{\bigcontribution_{1}}
\newcommand{\bigcontributiontwo}{\bigcontribution_{2}}
\newcommand{\bigcontributionthree}{\bigcontribution_{3}}
\newcommand{\bigcontributionshort}{\bigcontribution(\currenttime, \averagelengthshort, \price)}
\newcommand{\bigcontributionlong}{\bigcontribution(\currenttime, \averagelengthlong, \price)}
\newcommand{\bigcontributiontime}{\bigcontribution_\currenttime}
\newcommand{\averagelength}{L}
\newcommand{\averagelengthshort}{\averagelength_{1}}
\newcommand{\averagelengthlong}{\averagelength_{2}}
\newcommand{\kellyfraction}{F}
\newcommand{\rmserrorsquared}{\varepsilon^{{2}}_{rms_{\currenttime}}}
\newcommand{\rmserror}{\varepsilon_{rms_{\currenttime}}}
\newcommand{\amplitudecoefficientshort}{\amplitudecoefficientone}
\newcommand{\amplitudecoefficientlong}{\amplitudecoefficienttwo}
\newcommand{\genericfunction}{\zeta}
\newcommand{\scaleratiofactor}{\psi}
\newcommand{\Figure}[1]{Figure #1}
\newcommand{\tblofcontents}{\tableofcontents \pagebreak}
\newcommand{\portfolio}{V}

\titleformat*{\section}{\huge\bfseries}

\newcommand{\logo}
[0]
{\begin{figure}\centering\scalebox{0.25}{\includegraphics{logo.png}}\end{figure}}
\newcommand{\rulename}[1]
{\section*{Predictive Relationship: #1}}
\newcommand{\ruledescription}[1]{\hspace{200mm}\section{Trading Strategy Description}#1}
\newcommand{\relationshipdescription}[1]{\hspace{200mm}\section{Relationship Description}#1}

%Tried to make recursive function. Table lines are buggy
\makeatletter
\newcommand{\ruleparameters}{\section{Rule Parameters}
Below is a table summarizing the parameters specific to this trading rule.
\begin{center}
\begin{tabular}{|m{9em}| m{3em}| m{15em}| m{3em}|}
\hline
Parameter Name & Default Value &  Description & Symbol
\@tablefilli
}

\newcommand{\fixedruleparameters}{\section{Fixed Strategy Parameters}
Below is a table summarizing the parameters specific to this trading rule.
\begin{center}
\begin{tabular}{|m{9em}| m{3em}| m{15em}| m{3em}|}
\hline
Parameter Name & Default Value &  Description & Symbol
\@tablefilli
}
\newcommand\@tablefilli{\@ifnextchar\stoptable{\@tablesend}{\@tablefillii}}
\newcommand\@tablefillii[4]{%
  \@tablefilliii{#1}{#2}{#3}{#4}
  \@tablefilli % restart the recursion
}
\newcommand\@tablefilliii[4]{%
  \\ \hline
  #1 & #2 & #3 & #4
}
\newcommand\@tablesend[1]{% The argument is \stoptable
  \\ \hline
  \end{tabular}
  \end{center}}
\makeatother


\newcommand{\furtherlinks}[0]{
\hspace{200mm}
\hspace{200mm}
\section*{Further Links}
\begin{enumerate}
    \item InferTrade: \url{https://www.infertrade.com}
    \item Privacy Policy/Legal notice: \url{https://www.infertrade.com/privacy-policy}
    \item InferStat Ltd: \url{https://www.inferstat.com}
\end{enumerate}
}


\newcommand{\appendixinfertrade}[0]{
    \hspace{200mm}
    \hspace{200mm}
    \section*{Appendix}
}


\newcommand{\regressionanderrorpartone}{
    \subsection*{Calculating Regression Error}
    \vspace{1mm}
    \justify This PDF omits any explanation of how the regression coefficients $k_1$ and $k_2$ are calculated as we are using a standard regression for calculation of both the coefficients and standard error. For InferTrade we used the open source \textit{sklearn} Python library. The same can be achieved in Microsoft Excel by using the \textit{FORECAST.LINEAR} function.

    \vspace{1mm}
    \justify As a worked illustration of the error calculation, assume that InferTrade predicted that a security would increase in price by $0.1\%$ between yesterday and today, and the price actually decreased by $4.9\%$. In this example, $\Delta \price_{\currenttime} = +0.1\%$, and $\Delta\observed_{\currenttime} = -4.9\%$, giving $\Delta P_{\currenttime} - \Delta\price_{\currenttime} = 0.1\% - 4.9\% = -5\% = -0.05$. This value is then squared to give $(\Delta\price_{\currenttime} - \Delta\observed_{\currenttime})^{2} = (-0.05)^2 = 0.0025$. To obtain $\rmserrorsquared$ for day $\currenttime$, this calculation is repeated for every day from day $i = \currenttime-L$ to day $i = \currenttime$, with values summed before dividing by the regression length $\lookbacklength$.

}

\newcommand{\regressionanderrorparttwo}{
    \vspace{1mm}


\begin{equation}
\Delta \price_{\currenttime+1} = k_1 \research_\currenttime+k_2 + \epsilon_{\currenttime+1}
\tag{\ref{eqn:levelregression} }
\end{equation}

 As $\lookbacklength = 120$ for all simplified mode relationships, this is the same as squaring the root-mean-square of the previous 120 prediction errors .
    \vspace{1mm}

    \justify This section can be skipped or skimmed by those comfortable with implementing a standard rolling linear regression and a standard rolling error. As seen previously, InferTrade uses a standard linear regression over a historical rolling window of length $\lookbacklength$ to construct an estimator for daily price changes in the form:
    

    
    \vspace{1mm}
    \justify \textbf{NB} For the first $L$ days, $\Delta\price_{\currenttime+1}$ calculates the in-sample prediction, meaning that the one regression model is calculated over the first $L$ data points and used for predictions on each of these days. For any subsequent days, $\Delta\price_{\currenttime+1}$ calculates the out-of-sample prediction, meaning that a new regression model is fit for each trailing window. 
    
    \vspace{1mm}
    \justify While $\Delta\price_{\currenttime+1}$ is the regression estimator used to predict future price changes with an observed error $\epsilon_\currenttime$, the regression itself is calculated using the actual observed price changes $\Delta\observed_\currenttime$. The standard root-mean-squared error factor, which scales the suggested allocation on day $\currenttime$, is squared to obtain $\rmserrorsquared$ and is calculated in percentage terms as follows:
    
    \\ % creates some space after equation
    \vspace{1mm}
    \begin{equation}
        \rmserrorsquared = 
        \begin{cases}
        \frac{1}{\lookbacklength} \mathlarger{\sum}_{i=\currenttime-L}^{\currenttime}(\Delta \price_{i} - \Delta \observed_{i})^{2} & \text{if } \currenttime > L,\\
        0 & \text{otherwise}
        \end{cases}
    \end{equation}
    \vspace{1mm}
    \justify
    
    
    \justify where $\Delta\price_{i}$ and $\Delta\observed_{i}$ are the predicted and observed percentage price changes from day $i-1$ to day $i$, and $\lookbacklength$ is the regression length. 
    
    \\ % creates some space after equation
    \vspace{1mm}
    \justify For example, assume that InferTrade predicted that a security would increase in price by $0.1\%$ between yesterday and today, and the price actually decreased by $4.9\%$. In this example, $\Delta \price_{\currenttime} = +0.1\%$, and $\Delta\observed_{\currenttime} = -4.9\%$, giving $\Delta P_{\currenttime} - \Delta\price_{\currenttime} = 0.1\% - 4.9\% = -5\% = -0.05$. This value is then squared to give $(\Delta\price_{\currenttime} - \Delta\observed_{\currenttime})^{2} = (-0.05)^2 = 0.0025$. To obtain $\rmserrorsquared$ for day $\currenttime$, this calculation is repeated for every day from day $i = \currenttime-L$ to day $i = \currenttime$, with values summed before dividing by the regression length $\lookbacklength$. As $\lookbacklength = 120$ for all simplified mode relationships, this is the same as squaring the root-mean-square of the previous 120 prediction errors .
    \vspace{1mm}
    \justify Note that if the magnitude $|\rmserror| < 1$, dividing by $\rmserrorsquared$ increases the suggested allocation. This has the effect of suggesting higher allocations when the rolling error is small, thereby providing some adjustment for risk arising from volatility in the regression estimate. If, in the example above, InferTrade predicts that tomorrow's price will increase by $0.1\%$, and the root-mean-square of the previous 120 errors $(\Delta\price_{i} - \Delta\observed_{i})$ is found to be $5\%$, then $\rmserrorsquared = 0.05^2 = 0.0025$, and the suggested allocation $\position_\currenttime$ will be scaled by factor of $\frac{1}{0.0025} = 400$, giving
    
    \\ % creates some space after equation
    \vspace{1mm}
    \begin{equation}
    \position_\currenttime = \frac{E[\Delta \price_{\currenttime+1}]}{\rmserrorsquared} = \frac{0.001}{0.0025} = 0.4 = 40 \%
    \end{equation}
    
    \vspace{1mm}
    \justify $E[\Delta\price_{\currenttime+1}]$ should be read as "The Expected Value of $\Delta\price_{\currenttime+1}$" and signifies that the actual price change $\Delta o_{\currenttime+1}$ is likely to deviate from $\Delta\price_{\currenttime+1}$ with prediction error $\epsilon_{\currenttime+1}$.
}


\newcommand{\howtotrade}[2]{
\section{How to Trade}
In order to trade with the rules InferTrade provides, we calculate allocations for each day. We then allocate that fraction of our total portfolio value (cash and securities) to the market we are trading - to do this we buy or sell securities to reach the target allocation.
#2
\subsection*{How Allocation Determines Trade Size}
The allocation is the fractional amount of the portfolios value used to determine the size of the trading position. For example, if the allocation for Microsoft (MSFT) shares is 50\%, and we have \$100, we invest \$50 so that the value of held stock is the same as the value of held cash.
\subsection*{Rule Specific Trading Details}
#1}
\newcommand{\graphdir}[1]{../rule_graphs/#1}
\newcommand{\ruleperformance}[1]{
\section{Rule Performance Considerations}
#1
}
\newcommand{\assumptions}[0]{}
\newcommand{\keyterms}{
\section{Glossary}
\begin{itemize}
    \item \textbf{Bullish:} Positive outlook on the market. Expectation of positive returns.
    \item \textbf{Bearish:} Negative outlook on the market. Expectation of negative returns.
    \item \textbf{Allocation:} The allocation is the fractional amount of the portfolios value used to determine the size of the trading position.
    \item \textbf{Parameter:} Value used by the trading rule in the calculation for trading position
    \item \textbf{Trading Rule:} Strategy to determine when to buy, hold or sell a position.
\end{itemize}
}

%
\usepackage{tocloft}%
\renewcommand{\cfttoctitlefont}{\Large\bfseries}%
%
\begin{document}
\normalsize%
\logo%
\rulename{Price Trend or Price Reversion Relationship} %Argument is name of rule
\tableofcontents
\relationshipdescription{
\justify
\Centering\textbf{In a Price Trend or Reversion relationship, the percentage distance between the price and its historical mean directly affects future changes in the price:}

\begin{equation}
\mathlarger{
    \Delta P_{t+1} \propto -k \left(\frac{P_t}{\mu_P}-1\right)
    }
\end{equation}
\justify where ~$\Delta P_{t+1}$ represents the future price change, $P_t$ represents the current price of the market you are trading, $\mu_P$ represents the historical mean of the market, and $k$ is a scalar. The bracketed term calculates the percentage difference between $P_t$ and $\mu_P$. Notice that this relationship does not depend on a separate signal series, only on the price history of the market. Intuitively, this relationship can be thought of as a Price Trend or Price Reversion, depending on whether $k$ is positive or negative.
\vspace{1mm}
\justify
\textbf{For negative $k$:} when the price moves above its historical mean, it is good to buy the market, as the price will tend to drift higher. When the price moves below its historical mean, it is good to sell the market as the price will tend to drift lower. This relationship looks to capture momentum style trends.
\vspace{1mm}
\justify
\textbf{For positive $k$:} when the price moves above its historical mean, it is good to sell the market, as the price will tend to revert back down towards towards its average. When the price moves below the its historical mean, it is good to buy the market as the price will tend to revert back up to its average. This relationship looks to exploit cyclical ups-and-downs by buying low and selling high.
\vspace{1mm}
\justify
After finding the relationship which maximises risk adjusted returns, InferTrade runs tests for statistical significance to verify that the relationship gives a predictive edge. A predictive Price Trend relationship can be used to invest when the percentage gap between the price and its historical average is high versus the historical average of this gap, and sell when percentage gap between the price and its historical average is low versus the historical average of this gap. If the scalar \textit{k} is positive, inverting the Price series would make the previous statement hold, as this would become a predictive Price Reversion relationship.
\vspace{1mm}
\justify
It is difficult to identify types of time series likely to show a Price Trend relationship, as any price trending model is likely to be valid for only a temporary period after regressing. Similarly, it is difficult to identify types of time series likely to show a Price Reversion relationship, as markets tend to manifest a Brownian dynamic and typically move further away from their mean over longer periods of time due to changing fundamental factors.
}%

\ruledescription{
A Price Trend or Price Reversion relationship can be reflected in many kinds of rules. InferTrade uses a 120 period (6 months for daily data) rolling regression of the percentage gap between the price and its historical mean against next day's price change as a benchmark. \textbf{This trading rule recommends a constant portfolio allocation such that position size as a percentage of portfolio size will always stay the same}. The smaller the error, the greater the size of the recommended allocation. 
\vspace{1mm}
\justify
The allocation is scaled by the Kelly Fraction, a mathematically proven formula for determining optimal bet sizing. This rule will show higher returns than usual after optimisation if a significant price trend or price reversion relationship is present between the price its historical mean. The following equation shows how a Change Regression trading rule calculates a position sizing:

\begin{equation}
\position_\currenttime = \varphi \fixedconstant
\end{equation}
\\ % creates some space after equation
where $\position_\currenttime$ is the portfolio allocation at time $\currenttime$, $\varphi$ is the Kelly fraction and $\fixedconstant$ is a fixed constant.
\vspace{1mm}
\justify
\noindent The risk for the rule is that the market is not range bound, but trends to a high or low value. In these conditions constant position size will under-perform on a risk adjusted basis, as it held less and less securities when the price was rising and/or bought more and more securities as the price continued to fall - both would under-perform holding a fixed number of securities.\\

\noindent If the position allocation is negative then risk-adjusted performance will be the other way around (momentum following), as a -50\% allocation to shares will increase the number of short securities as the market falls. This would be very profitable if the market trends (e.g. share heads to bankruptcy), but will under-perform a constant short if the market see-saws in price and/or is range bound.\\

\noindent The rule is also useful as a benchmark as rules with sticky allocations (that tend to invest a similar \% over consecutive days), will tend to perform well in range trading conditions even if they are not predictive of the market. Subsequently a good trading strategy should derive returns in excess of constant position size as well as outperforming its underlying market.\\
} %Argument is rule description

{
\ruleparameters%You can include however many arguments (in groups of 4) as you want!
{Constant Position Size}{0.5}{This is the fraction of the portfolio's value to invest in the position, before scaling by the Kelly Factor. Valid values range from 0 to 1.}{$\fixedconstant$}
{Amplitude}{0.1}{Amplitude weighting (Kelly Fraction). 1.0 is maximum growth if regression is exact. <1.0 scales down positions taken.}{$\varphi$}
\stoptable %must be included or Tex engine runs infinitely

\keyterms
\furtherlinks %The footer
\end{document}

